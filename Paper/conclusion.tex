\section{Conclusion}

To summarize, in my research I created an agent-based model that allowed me to explore the spread of political polarization in a society. Through my cross-issue influence mechanism, I find that network density is negatively correlated with assortativity. Additionally, I find that when individuals in a society are less receptive to differing opinions, there will be less consensus for any given issue in that society. Finally, and potentially most important, I conclude that the cross-issue influence mechanism is sufficient in generating issue alignment endogenously in a society. 

In addition to the findings of my model, I learned many valuable skills that I will take with me beyond my time at the University of Mary Washington. 

Aside from the sheer amount of python programming that this project required, I have also strengthened my data analysis skills through analyzing the results of various parameter sweeps. To ensure the robustness of my conclusions, I have explored the results graphically and through statistical methods. 

Finally, this project has given me the useful ability of creating agent-based models. Increasingly, agent-based models are being used to replicate complex systems in areas where data is not readily available such as fraud detection and military simulation. The data from an agent-based model is then able to be combined with machine learning to create predictive algorithms that are profitable for modern companies. 

In my professional career beyond the University of Mary Washington, I plan on continuing to utilize the skills and knowledge gained from my two semesters of research experience.   