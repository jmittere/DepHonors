\begin{abstract}
I present an agent-based model, inspired by the opinion dynamics (OD) literature, to explore the underlying behaviors that may induce societal polarization. My agents interact on a social network, in which adjacent nodes can influence each other, and each agent holds an array of continuous opinion values (on a 0-1 scale) on a number of separate issues. I use three measures as a proxy for the virtual society's ``polarization:'' the average assortativity of the graph with respect to the agents' opinions, the number of distinct opinion buckets in which agents have the same opinions after the model reaches an equilibrium, and the number of non-uniform issues.

I look at multiple model parameters that affect polarization. The first is the density of edges in the network: this corresponds to the average number of meaningful social connections that agents in a society have. Contrary to my early hypothesis, I find that lower edge density results in higher levels of assortativity for Erd\"{o}s-R\'{e}nyi
graphs. The second is the level of ``openness'' and ``disgust'' of agents to differing opinions; \textit{i.e.}, how close or distant a neighboring node's opinion on an issue must be to an agent's own before the agent will adjust its opinion on a different issue. I refer to this novel mechanism as cross-issue influence. Through this mechanism, I find that when agents in the model are less open to new opinions, there will be less consensus on any given issue for all agents in the model. Additionally, I find that there will be fewer distinct opinion buckets and therefore higher polarization in models where agents follow a cross-issue influence mechanism compared to same-issue influence.   
\end{abstract}
