\begin{abstract}
We present an agent-based model, inspired by the opinion dynamics (OD) literature, to explore the underlying behaviors that may induce societal polarization. Our agents interact on a social network, in which adjacent nodes can influence each other, and each agent holds an array of continuous opinion values (on a 0-1 scale) on a number of separate issues. We use two measures as a proxy for the virtual society's ``polarization:'' the average assortativity of the graph with respect to the agents' opinions, and the number of issues on which agents have persistent disagreement even after the model reaches an equilibrium.

We look at two model parameters that affect polarization. The first is the density of edges in the network: this corresponds to the average number of meaningful social connections that agents in a society have. Contrary to our early hypothesis, we find that lower edge density results in higher levels of assortativity. The second is the ``openness'' of agents to differing opinions; \textit{i.e.}, how close a neighboring node's opinion on an issue must be to an agent's own before the agent will adjust its opinion on a different issue. We refer to this novel mechanism as cross-issue influence. Through this mechanism, we find that when agents in the model are less open to new opinions, there will be less consensus on any given issue for all agents in the model.   
\end{abstract}
