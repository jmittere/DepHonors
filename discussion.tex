\section{Discussion and Future Work}

Multiple results in this research surprised me. Firstly, the results from
testing $H_{1a}$ did not reflect my anecdotal experiences. When increasing the
density of a society's connection, I instead saw \textit{lower} assortativity.
I believe this may be due to the static nature of the model's social network.
In the real world, homophily not only causes existing friends to become more
like each other, but also causes people to select (or reject) friends based on
their similarity. In future work, I intend to add this feature to the model,
producing a dynamic graph, and discover whether this addition is sufficient to
produce a positive density/assortativity relationship.

The lack of a relationship for $H_{1b}$ was another surprising result. I
extensively tested this hypothesis, but the results did not indicate any
statistically significant relationship. This result remains unexplained.

The tipping points observed when testing hypotheses $H_{2a}$ and $H_{2b}$ were
compelling results. When even slightly increasing the density of a graph, the
number of clustered issues can drop quickly. This would seem to indicate that
the degree to which a society forms consensus can be quite sensitive to the
average number of social connections people maintain, at least within a certain
range. Too, the openness of a society's members -- however that might be
quantified in a real population -- produced an even steeper tipping point. One
interpretation would be that even small changes in the tolerance people have
for dissenting views can produce great gains in reducing polarization. I also
plan to investigate the behavior of models with agents that are heterogeneous 
with respect to openness, since OE and other traits are obviously not uniform
across a real population.

Finally, the most exciting results in my opinion were the findings of hypothesis $H_{3}$. This result could have wide-reaching implications on how we interpret the causes of the polarization phenomenon, and what societal changes might be necessary to reduce it. First, this finding shows that neither ideological coherence nor media influence is necessary for issue alignment to develop in a society. It shows that the cross-issue influence mechanism is sufficient. Secondly, issue alignment endogenously appears through the CI2 mechanism. The model was not initialized with any interdependencies between issues, yet this phenomenon develops consistently through CI2. 

Although this is an exciting result, it is hard to be joyful about replicating a negative phenomenon such as polarization. However, replicating polarization has left me with certain ideas about how to reduce polarization from the individual and societal perspective. First, relating to the finding that lower network density leads to higher polarization, I believe that a potential fix for individuals is to surround yourself with more people. By increasing your number of social connections, there is a higher chance you will be exposed to a more diverse set of opinions; thus, lowering your chance of being in an echo chamber. Second, relating to the openness findings, I believe that individuals can help to reduce societal polarization by increasing their tolerance level to dissenting opinions. Finally, relating to the issue alignment findings from $H_{3}$, I believe the solution is to allow yourself the freedom to agree with someone on one thing and disagree with them on something else. By keeping unrelated issues separate, the cross-issue influence mechanism can be reduced which would decrease the degree of issue alignment in society. From a societal perspective, to reduce the issue alignment we need voices from more buckets. The two party system and global media outlets undoubtedly reinforce the issue alignment phenomenon. The solution on a macro level is larger variety of reasonable opinions.    
