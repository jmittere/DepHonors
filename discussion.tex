\section{Discussion and Future Work}

Multiple results in this research surprised us. Firstly, the results from
testing $H_{1a}$ did not reflect our anecdotal experiences. When increasing the
density of a society's connection, we instead saw \textit{lower} assortativity.
We believe this may be due to the static nature of the model's social network.
In the real world, homophily not only causes existing friends to become more
like each other, but also causes people to select (or reject) friends based on
their similarity. In future work, we intend to add this feature to the model,
producing a dynamic graph, and discover whether this addition is sufficient to
produce a positive density/assortativity relationship.

The lack of a relationship for $H_{1b}$ was another surprising result. We
extensively tested this hypothesis, but the results did not indicate any
statistically significant relationship. This result remains unexplained.

The tipping points observed when testing hypotheses $H_{2a}$ and $H_{2b}$ were
compelling results. When even slightly increasing the density of a graph, the
number of clustered issues can drop quickly. This would seem to indicate that
the degree to which a society forms consensus can be quite sensitive to the
average number of social connections people maintain, at least within a certain
range. Too, the openness of a society's members -- however that might be
quantified in a real population -- produced an even steeper tipping point. One
interpretation would be that even small changes in the tolerance people have
for dissenting views can produce great gains in reducing polarization. We also
plan to investigate the behavior of models with agents that are heterogeneous 
with respect to openness, since OE and other traits are obviously not uniform
across a real population.

Another mechanism we hope to explore more in future research is cross-issue
influence. This concept is an extension of Hegselmann and Krause's
bounded-confidence mechanism. In this research we explore cross-issue influence
with only attracting forces. However, we hope to investigate the results that
would be produced when a repelling force is implemented into the cross-issue
influence mechanism. Rather than only having agents move closer to one another
on issue X, we could also have them be pushed away from each other on issue X
if they disagree above a certain threshold on a separate issue Y.    
