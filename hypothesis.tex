\section{Hypothesis}
I form the following hypotheses about the model's behavior.

\textbf{Hypothesis 1a}: ($H_{1a}$). Mean assortativity will increase with the
edge probability of an Erd\"{o}s-R\'{e}nyi graph.

\textbf{Hypothesis 1b}: ($H_{1b}$). Mean assortativity will increase when the
openness threshold of agents in the model is lower.

\textbf{Hypothesis 2a}: ($H_{2a}$). The number of clustered issues will be
negatively correlated with the edge probability of an Erd\"{o}s-R\'{e}nyi
graph.

\textbf{Hypothesis 2b}: ($H_{2b}$). The number of clustered issues will
increase when the openness threshold is lower for all agents in the model.

\textbf{Hypothesis 3}: ($H_{3}$). The number of distinct opinion buckets will
decrease when the agents in the model follow the cross-issue influence mechanism compared to same-issue influence mechanism.


For $H_{1a}$, I hypothesize that increasing the connectivity of an
Erd\"o{s}-R\'{e}nyi graph by raising the edge probability will result in higher
assortativity. This hypothesis is based mainly on real-world observations: the
number of social connections available to those with Internet access has
increased in the past few decades (due to social media\cite{dean_how_2021}),
and the degree of homophily exhibited in members of a social circle has also
(at least anecdotally) increased. Since both the density of connections and the
homophily of those joined by such connections has increased in the real world,
I presume the same effect will follow in my model.

For $H_{1b}$, I hypothesize that when agents in the model are less open to new
opinions, there will be a higher average assortativity and therefore
polarization. When all agents have a lower level of openness, they will only be
interacting with agents that have opinions similar to their own; therefore, I
expect to see higher levels of assortativity. 

For $H_{2a}$, I assume that raising the connectivity of an Erd\"o{s}-R\'{e}nyi
graph by increasing the edge probability will result in fewer clustered issues.
As a graph becomes more densely connected, agents will have a wider variety of
neighbors to receive influence from. As a result, agents should merge to the
consensus opinion for any given issue more often in a more densely connected
graph. 

For $H_{2b}$, I believe that lowering the openness threshold of agents in the
model will result in more clustered issues across the model. When agents are
less open to distant opinions, there will be more variety of opinion for any
given issue. 

For $H_{3}$, I believe that when agents follow the cross-issue influence mechanism, there will be fewer opinion buckets because agents will converge to a few distinct sets of opinions. With the same-issue influence mechanism, there will be less convergence to sets of opinions, and therefore more distinct opinion buckets. 