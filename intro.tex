\section{Introduction}

The recent events that transpired at the U.S.~Capitol on January 6th were a
vivid reminder of the deep divide within the nation. There are signs that the
United States is experiencing political polarization now like it has never seen
before. As individuals stormed the Capitol, Americans watched in horror.
Although this singular event is now in the past, the underlying tension that
preceded it still remains.

Polarization -- reflected in echo chambers, entrenched views, and the
vilification of those whose opinion differs -- can be harmful to a democratic
society. It can inhibit the reaching of consensus and compromise upon which a
democracy is built, and can result in even greater amounts of damage than what
ensued in the U.S.~on January 6th if left unchecked. Further, polarization
affects not only political actors, but also the interpersonal relationships
among the rank and file citizens of a country which bolster and strengthen
society.

In this paper, we look at two societal variables that we believe may
significantly impact polarization in a society. The first is the density of
social connections: in other words, the average number of social ties a member
of that society has. The second is the degree of ``openness'' in the society:
namely, how willing its members are to consider changing their views. We
suspect that both of these factors play a role in determining the aggregate
polarization of a society.

In order to explore these phenomena, we created an Agent-Based Model (ABM) of
heterogeneous agents in the spirit of much of the Opinion Dynamics (OD)
literature. These agents interact with each other on a random, static social
network and change their opinions on issues over time based on the opinions of
their network neighbors. One novel feature of our model, termed cross-issue influence,
is the way agents influence one another: one agent will not allow another agent to influence its
opinion on an issue unless the two agents already have sufficient agreement on
another (randomly chosen) issue. The justification for this is related to the
well-known observation of ``homophily'' in social psychology: people are prone
to trust those who already agree with them on something, and hence are more
likely to be persuaded by them on other matters.

The goal of our research is to determine what micro behaviors of individuals
are sufficient to produce a change in the degree of political polarization in
the society. As explained below, we choose to measure polarization in two
different ways: the average similarity of an agent to its neighbors (called
``assortativity'' in social network terminology), and the likelihood that no
consensus will be reached on an issue (called opinion ``clustering'' in the OD
literature).
