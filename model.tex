\section{Model}

The model is presented using an abbreviated version of the ODD
protocol\cite{grimm_standard_2006}.

\subsection{Purpose}

% Make this compatible with definitions in Polarization lit review

The model simulates interactions on a random social network of agents, each
with an array of continuous, numeric \textbf{opinion} attributes. Its purpose
is to investigate the way in which multiple factors contribute to the emergence of
polarization in the network: the \textbf{edge\_probability}, a value reflecting
the density of social connections in the network; the \textbf{openness} threshold, a
value representing how closely one of an agent's opinions must be to that of a
potential influencer in order to accept influence; the \textbf{disgust} threshold, a value representing how far away one of an agent's opinions must be to that of a potential influencer in order to be repelled away on an issue; and the presence of agents following the cross-issue influence mechanism or the same-issue influence mechanism. (See
Section~\ref{modelProcess}, below.)

Using the model, I hope to gain general insight on the emergence of this
polarization within social networks and how different parameters affect this.

\subsection{Entities, State Variables and Scales}

The entities within the model are agents, having the following attributes:

\begin{description}
\item[ID] A unique ID for the agent.
\item[Opinions] An array of numbers, representing opinions on issues, each
having a value between 0 and 1. This represents the degree to which the agent
``agrees'' or ``disagrees'' with an issue, with 0.5 being neutral.
\item[Neighbors] A subset of the other agents in the model, to whom this Agent
has a social connection. The entire set of agents and their social connections
form an undirected graph (\textit{i.e.}, all social connections are
bidirectional) and the graph is fixed throughout the simulation.

\end{description}


\subsection{Process Overview and Scheduling}
\label{modelProcess}

After the model has been initialized, the following sequence is executed for
each of a fixed number of steps in the simulation for cross-issue influence agents:

\begin{compactenum}
\item An agent $X$ is chosen at random.  
\item A neighbor of $X$ (call it $Y$) is chosen at random.
\item An issue $I_1$ is chosen at random.
\item The absolute difference between $X$'s opinion on $I_1$ and Y's opinion on
$I_1$ is measured.
\item Another opinion $I_2$ ($\neq I_1$) is chosen at random.
\item If the difference between $X$'s and $Y$'s opinion on issue $I_1$ is less
than or equal to the model's \textbf{openness} threshold, set $X$'s opinion on
$I_2$ to be the average of $X$'s and $Y$'s current $I_2$ opinions.
\item If the difference between $X$'s and $Y$'s opinion on issue $I_1$ is greater
than or equal to the model's \textbf{disgust} threshold, calculate the difference between $X$'s opinion and the average of $X$ and $Y$'s current $I_2$ opinions. Then, add this quantity to $X$'s opinion on $I_2$. Note that for the cross-issue influence mechanism, $X$'s opinion will move away from $Y$'s on $I_2$. \\ 
\end{compactenum} 

For same-issue influence agents, the following sequence is executed:
\begin{compactenum}
\item An agent $X$ is chosen at random.  
\item A neighbor of $X$ (call it $Y$) is chosen at random.
\item An issue $I_1$ is chosen at random.
\item The absolute difference between $X$'s opinion on $I_1$ and Y's opinion on
$I_1$ is measured.
\item If the difference between $X$'s and $Y$'s opinion on issue $I_1$ is less
than or equal to the model's \textbf{openness} threshold, set $X$'s opinion on
$I_1$ to be the average of $X$'s and $Y$'s current $I_1$ opinions.
\item If the difference between $X$'s and $Y$'s opinion on issue $I_1$ is greater
than or equal to the model's \textbf{disgust} threshold, calculate the difference between $X$'s opinion and the average of $X$ and $Y$'s current $I_1$ opinions. Then, add this quantity to $X$'s opinion on $I_1$. Note that for the same-issue influence mechanism, $X$'s opinion will move away from $Y$'s on $I_1$. 
\end{compactenum}

\subsection{Initialization}

The simulation is initialized with 100 agents, each having a variable number of opinions set to
independent uniform random values between 0 and 1. The model is initialized with all agents either following the cross-issue influence mechanism or the same-issue influence mechanism. The agents are then
connected to each other using a random undirected Erd\"{o}s-R\'{e}nyi
graph\cite{erdos_random_1959} with parameters\ $N=100,
p=\textbf{edge\_probability}$. If the graph is not connected, a new random
graph is generated until a connected one is obtained.

An openness threshold and disgust threshold, each having a value between 0 and 1, will be set such that the openness threshold is always less than the disgust threshold.

The model's step limit is usually set to 1000, as most change in the agent's
opinions after 1000 steps is negligible.

