\section{Psychological Foundations}

\subsection{Openness and Disgust}
Regarding the role of a society's ``openness'' and ``disgust,'' one question that arises is the
psychological basis for these attributes. Which personality trait plays the
biggest role in an individual's likelihood to change their opinions on a
particular issue? 

The `Big 5' personality trait group\cite{john_big-five_1999},
well-researched since the 1980s, contains Openness-to-Experience (OE) as one of
its five traits. OE can be defined as ``cognitive
flexibility''\cite{deyoung_sources_2005}, or ``[openness can be] associated
with having a vivid imagination and [...] receptivity to one's own and other's
emotions; a willingness to try new experiences''\cite{furnham_childhood_2016}.

As the research shows, openness plays a crucial role in an individual's ability
to relate to others, as well as to consider adopting outside ideas as their
own. On the other side of openness is disgust. More than the Big 5's Agreeableness and Conscientiousness traits, OE seems
to encompass openness and disgust in my model.

\subsection{Homophily}
Homophily is a well-proven principle in social psychology that simply says when individuals are more similar, they are more likely to trust each other\cite{mcpherson_birds_2001}. As a result, they are more likely to be influenced by those like-minded individuals because they already trust them. This principle is a large basis for both same-issue influence and cross-issue influence in my model. 