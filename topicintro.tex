\section{Topic Introduction}

\subsection{Modeling and Simulation}
First, modeling and simulation is the large field that my research falls under. Modeling and simulation is a rapidly expanding field where the goal is to model an environment to explore phenomena or effects that occur (or will occur) in the real-world. Models can be made to explain or predict real-world behavior. 

In the field of modeling and simulation, professionals in many disciplines have used methods to simulate complex real-world systems. For example, many people benefit from these types of simulations every day when they check weather applications. Weather apps take in large amounts of data from live feeds in the environment. Then, researchers use the data to create a simulated environment where they can observe and predict future weather outcomes. 


Another area where this type of model is used is in the predictions of stock prices. Economists are able to input large amounts of data into a model which can predict the direction of stock prices. These types of simulations that are created to be as close to real-world systems as possible are called facsimiles. They are extremely complex and require large amounts of data to retain their predictive power. The models that I wish to do research on are different from these complex and costly models. My goal is not to create a one-to-one model of the United States in 2010 and see if political polarization develops by 2021. This would not be feasible as I would need to model the United States in 2010 which is an impossible task. Instead, I create a simulation where agents (who represent people) interact on a social network and slowly change the views of those around them in what is called an \textbf{Agent-Based Model}.  

\subsection{Modeling Techniques}
Within the field of modeling and simulation, there are many techniques that have been used to model complex systems. One such technique is discrete event simulation (or DES). In DES, states are modeled as atomic representations of the environment. For example, queuing theory is often represented with DES. Using DES, we gain valuable insights such as predictions on the average length of a queue and the potential longest wait time. Another modeling technique is system dynamics (or SD). SD is a technique that involves using differential equations to model a system. This technique is widely used to model continuous relationships in studying things like epidemiology, population growth, industrial economics, and economic policy.

\subsection{Agent-Based Modeling}
Although DES and SD are useful modeling techniques, they are not applicable for every complex system. These traditional modeling techniques are useful in the aggregate; however, they often fail to capture the impact of agent heterogeneity. For example, macroeconomic models assume that all agents are homogeneous which allows for a simpler model. Such models allow for conclusions about the equilibrium conditions of an economy. Although these equilibrium conclusions are helpful for policy decisions, homogeneous agents limit the complexity and authenticity of a simulation of human behavior because real people are neither carbon copies of one another, nor totally deterministic in their relationships and choices. Given my research topic of political polarization, a DES or SD model would not be the right approach. The method that I use is \textbf{agent-based modeling} which is commonly called an ABM for agent-based model. ABMs allow agents to have their own behavior according to characteristics that are unique to that particular agent which is extremely beneficial in simulating a social network. \\

Agent-Based modeling is a technique that is growing in popularity. Historically, agent-based modeling has not been a widely used approach in the field of modeling and simulation due to computing power restrictions. Large agent-based models generally suffer from the issues of high space and time complexity. In the past, agent-based models were even simulated by researchers calculating mathematical results by hand without the aid of computer programs. With 21st century technology, we are able to simulate large systems using ABMs of heterogeneous agents. 

Some common uses of ABMs are in the study of social and cultural phenomena in economics, demography, and sociology. Additionally, ABMs are used to study natural phenomenon in fields like biology and epidemiology. With the Covid-19 pandemic, the popularity of ABMs have soared due to their ability to model disease transmission through a population where individuals respond differently to policy decisions like lock downs and mask mandates and to health offerings like vaccines. 

Using an ABM, my aim is "...to 'grow' certain social structures in the computer...the aim being to discover fundamental local or micro mechanisms that are sufficient to \textit{generate} the macroscopic social structures and collective behaviors of interest"(Joshua Epstein in Growing Artificial Societies **need citation). I attempt to discover the micro mechanisms that are sufficient to produce the social and cultural phenomenon of political polarization. I pull from other disciplines such as sociology and psychology for theories of human behavior that inspire the influence mechanisms of my agents. These influence mechanisms will be implemented on the micro level such that agents follow behavior specified by me when I initialize the model. I also use various computer science and statistics techniques for implementation and analysis of the macro level trends that occur in the model.    

In addition to the large benefit of being able to explore macro level trends accurately, ABMs are also a practical approach to modeling due to the simple nature of their code implementation. Those that are familiar with the modern object-oriented programming approach (or OOP) may already see the connection between ABMs and OOP. In OOP, objects have instance variables and functions that are unique to that class. Objects are often contained in data structures (which may even be an instance variable of another class). 

Using OOP, I can create an agent class. The agent class may contain instance variables that are useful in modeling such as the age, wealth, location, and neighbors (other connections in the social network) of that particular agent object. These agent objects represent entities that exist within the physical or abstract environment of the model. In my case, the agents will represent people, but its possible for agents to be animals, companies, or even simple biological organisms. Oftentimes, an ABM also has a model class that follows the singleton design pattern. This model class generally contains many agents in a data structure. The encapsulation in OOP greatly aids in the development of a large ABM. With the encapsulation of agent objects in a model object, I am able to create model-level functions that provide high-level insights into the behavior of the agent objects. Using an ABM, I will study the \textbf{opinion dynamics} of an artificial society that interacts on a social network.


\subsection{Opinion Dynamics}
Opinion dynamics is the study of how opinions spread throughout a society. Agent-based models are especially useful for studying opinion dynamics because researchers are able to design agents and tailor their interactions. Researchers are able to test the consequences on a society of specific agent behaviors by studying how opinions flow from agent to agent. In my case, I will be designing agents to follow two different influence mechanisms. Then, I will observe how opinions flow throughout the society and the polarization that develops as a result of each respective influence mechanism.

\subsection{Modeling Terminology}

\textbf{Parameter Suite} and \textbf{Parameter Sweep}. To get robust results when investigating my hypothesis, I ran parameter suites which are batches of models that are run with the same values for each parameter in the model. With results from a parameter suite, I am able to minimize the impact of randomness and outliers which helps me determine the average result of the model with a certain set of parameters. Another useful modeling technique is the parameter sweep. Parameter sweeps allow me to vary one (or multiple) parameter(s) of the model to see how the output of the simulation varies with the parameter. Parameter sweeps consist of many parameter suites. It is helpful to think of parameter suites like playing a game against an opponent \textit{X} times to find the true outcome of the game. Parameter sweeps are like playing against every opponent \textit{X} times to capture the full impact of varying the opponent on your average outcome when playing the game.\\

\textbf{Social Network}. One term that I will use to describe the entirety of my model is a social network. In this instance, I do not mean a social network like Facebook or Twitter, but rather, a social network is a group of collected nodes linked by edges that influence the opinions of one another. In my model, the nodes represent agents. Additionally, in my model the edges will be undirected. It is possible and even common for ABMs to have directed edges. In models where researchers are trying to simulate a follower-relationship like Twitter, edges are often directed. However, in my model, edges are meant to represent friendships that are mutual between two agents. If you are ``friends'' with another agent, you can influence them, and they can influence you. As a result, agents can influence and be influenced by their neighbors (nodes they are directly connected to on the graph). 

\subsection{Technologies Used}
To conduct my research I used the Python programming language due to the wide array of libraries and packages that are available. 

One important package that I used was Mesa. Mesa is an agent-based modeling framework for Python that allows for flexibility in modeling decisions while still providing boilerplate code and a pre-built scheduler for agents. Additionally, Mesa provides a datacollector object that allowed me to analyze the model down to the agent level and more broadly at the model level. 

In addition to Mesa, I used NetworkX which is a package that allows for easy creation, manipulation, and analysis of the structure and dynamics of a graph. With NetworkX, I was able to generate different random connected graphs by using different graph generation algorithms. I used NetworkX to generate Erd\"{o}s-R\'{e}nyi graphs, Watts-Strogatz graphs (also known as small-world networks), and Barab\'{a}si-Albert graphs (also known as preferential-attachment networks). For the purposes of this paper, I will only be presenting results that were found on Erd\"{o}s-R\'{e}nyi graphs. 

Another package that I considered using for clustering in the model is AutoGMM. AutoGMM offers an algorithm for automatic Gaussian mixture modeling that I was contemplating using for the clustering across issues. However, estimating the optimal model and number of clusters is an NP-Hard problem, so the time complexity of using the package prohibited me from implementing it. 

Other packages used were Numpy and Scipy for mathematical operations and Pandas for data analysis. 