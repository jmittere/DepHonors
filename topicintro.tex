\section{Topic Introduction}

\subsection{Modeling and Simulation}
First, modeling and simulation is the large field that my research falls under. Modeling and simulation is a rapidly expanding field where the goal is to model an environment to explore phenomena or effects that occur (or will occur) in the real-world. Models can be made to explain or predict real-world behavior. In the field of modeling and simulation, professionals in many disciplines have used methods to simulate complex real-world systems. For example, many people benefit from these types of simulations every day when they check weather applications. Weather apps take in large amounts of data from live feeds in the environment. Then, researchers use the data to create a simulated environment where they can observe and predict future weather outcomes. Another area where this type of model is used is in the predictions of stock prices. Economists are able to input large amounts of data into a model which can predict the direction of stock prices. These types of simulations that are created to be as close to real-world systems as possible are called facsimiles. They are extremely complex and require large amounts of data to retain their predictive power. The models that I wish to do research on are different from these complex and costly models. My goal is not to create a one-to-one model of the United States in 2010 and see if political polarization develops by 2021. This would not be feasible as I would need to model the United States in 2010 which is an impossible task. Instead, I will create a simulation where agents (who represent people) interact on a social network and slowly change the views of those around them in what is called an \textbf{Agent-Based Model}.  

\subsection{Modeling Techniques}
Within the field of modeling and simulation, there are many techniques that have been used to model complex systems. One such technique is discrete event simulation (or DES). In DES, states are modeled as atomic representations of the environment. For example, travel can be modeled with DES. A traveler starts in one location \textit{A}, flys to another location \textit{B}, and then drives to the final location \textit{C}. Another modeling technique is system dynamics (or SD). SD is a more calculus-centered approach. These models are often used to measure continous relationships between entities in fields like industrial economics, environmental policy, and demography.

\subsection{Agent-Based Modeling}
Although DES and SD are useful modeling techniques, they are not applicable for every complex system. These traditional modeling techniques are useful in the aggregate; however, they often fail to capture the impact of agent heterogeneity. For example, macroeconomic models assume that all agents are homogenous which allows for a simpler model. Such models allow for conclusions about the equilibrium conditions of an economy. Although these equilibrium conclusions are helpful for policy decisions, homogenous agents limit the complexity and authenticity of a simulation of human behavior because real people are neither carbon copies of one another, nor totally deterministic in their relationships and choices. Given my research topic of political polarization, a DES or SD model would not be the right approach. The method that I use is \textbf{agent-based modeling} which is commonly called an ABM for agent-based model. ABMs allow agents to have their own behavior according to characteristics that are unique to that particular agent which is extremely beneficial in simulating a social network. \\

Agent-Based modeling is a technique that is growing in popularity. Historically, agent-based modeling has not been a widely used approach in the field of modeling and simulation due to computing power restrictions. Large agent-based models generally suffer from the issues of high space and time complexity. In the past, agent-based models were even simulated by researchers calculating mathematical results by hand without the aid of computer programs. With 21st century technology, we are able to simulate large systems using ABMs of heterogenous agents. 

Some common uses of ABMs are in the study of social and cultural phenomena in economics, demography, and sociology. Additionally, ABMs are used to study natural phenomenon in fields like biology and epidemiology. With the Covid-19 pandemic, the popularity of ABMs have soared due to their ability to model disease transmission through a population where individuals respond differently to policy decisions like lockdowns and mask mandates and to health offerings like vaccines. Using an ABM, my aim is "...to 'grow' certain social structures in the computer...the aim being to discover fundamental local or micro mechanisms that are sufficient to \textit{generate} the macroscopic social structures and collective behaviors of interest"(Joshua Epstein in Growing Artificial Societies **need citation). I attempt to discover the micro mechanisms that are sufficient to produce the social and cultural phenomenon of political polarization. I pull from other disciplines such as sociology and physcology for theories of human behavior that I will implement on the micro level. I use various computer science and statistics techniques for implementation and analysis of the macro level phenomena that occur in the model.    

In addition to the large benefit of being able to explore macro level phenomena accurately, ABMs are also a practical approach to modeling due to the simple nature of their implementation. Those that are familiar with the modern object-oriented programming approach (or OOP) may already see the connection between ABMs and OOP. In OOP, objects have instance variables and functions that are unique to that object. Objects are often contained in data structures (which may even be an instance variable of another class). Using OOP, we can create an agent class. The agent class may contain instance variables that are useful in modeling such as the age, wealth, location, and neighbors (other agents in the model) of that particular agent object. These agent objects represent entities that exist within the physical or abstract environment of our model. In my case, the agents will represent people in my model, but in other models, agents can be animals, companies, or biological organisms. Often times, an ABM also has a model class that follows the singleton design pattern. This model class generally contains many agents in a data structure. The encapsulation in OOP greatly aids in the development of a large ABM. With the encapsulation of agent objects in a model object, we are able to create model-level functions that provide high-level insights into the behavior of the agent objects. Using an ABM, I will study the \textbf{opinion dynamics} of an artificial society that interacts on a social network.


\subsection{Opinion Dynamics}
