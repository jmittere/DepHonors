
\section{Variables}

In this section we define the two important independent variables whose effect
on the model's behavior we seek to discover, and the two dependent variables we
measure at simulation's end.

\subsection{Independent variables}

\subsubsection{Openness}

As mentioned earlier, research shows that \textbf{openness} plays a crucial role in an
individual's ability to relate to others, as well how easily they adopt outside
ideas as their own. To quantify this as a model parameter, we incorporate
openness as a threshold on a continuum from 0 to 1; this threshold is used to
compare agent opinions during their pairwise interactions. Low levels of
openness produce models in which agents only very rarely change their opinions
(namely, only when encountering neighboring agents whose opinion on another
issue is very close to their own). High levels produce models in which agents
eagerly incorporate the opinions of others on almost every interaction.

\subsubsection{Edge probability}

The other parameter represented in our model is the \textit{density} of social
connections. To implement the concept of different degrees of social
connection, we used the Erd\"{o}s-R\'{e}nyi graph generation algorithm to
generate a random graph of connected nodes. With the Erd\"{o}s-R\'{e}nyi graph
generation algorithm, we can specify the \textbf{edge probability} which represents the
probability that there will be an edge between any two given nodes. Using the
edge probability, we can control the density of the resulting graph. As a
result, edge probability directly corresponds to the density of social
connections in our model.    

\subsection{Dependent variables}


\subsubsection{Graph assortativity}

One way we measure the simulated society's polarization is through the
resulting network's ``assortative mixing,'' or simply graph
\textbf{assortativity}. This represents the degree to which an agent's opinions
will have similar values to those of its network neighbors, on average.

The assortativity of a network has a value between $-1$ and 1, where 1
indicates ``perfect assortative mixing'' -- \textit{i.e.}, a situation where
every agent's opinions are identical to each of its graph neighbors'. An
assortativity of 0 indicates that the agents' social connections have no
correlation at all with their opinion values: having a social tie with another
agent does not mean an agent is any more (or less) likely to have opinions
similar to that agent. This will be approximately true when the model is
initialized and before the iterative process begins. (Negative assortativity
values correspond to networks in which an agent is \textit{less} likely to
agree with its network neighbors than with agents in general.)

Assortativity is thus a way to measure the extent to which agents become
surrounded by (only) like-minded agents, and are therefore no longer exposed to
alternative points of view. Since we need to obtain the graph's assortativity
with respect to \textit{multiple} attributes (\textit{i.e.}, the opinions an
agent has on all of the issues), we simply compute the network's assortativity
for each issue separately (as defined in \cite{newman_mixing_2003}, p.5) and
average it over all the issues.

\subsubsection{Opinion clustering}

The second dependent variable of our model is opinion \textbf{clustering}. This
measures how often the opinions that agents have on a given issue fail to
converge to a uniform value, instead remaining bifurcated among two or more
values in perpetuity. Each group of agents who, at simulation's end, have the
same opinion on an issue (within some small tolerance $\epsilon$) are termed an
``opinion cluster'' (a term used by \cite{fotakis_opinion_2016}) on that issue.

For clarity, we refer to any issue on which all agent opinions eventually
converge to the same value as a ``\textbf{uniform issue},'' and any issue that
instead produces opinion clusters as a ``\textbf{clustered issue}.''

One challenge is defining what qualifies as an clustered issue, given that
agent opinions are represented as real numbers that may asymptotically converge
to, but never actually reach, the same value. We use the following mechanism.
To calculate the number of clusters for an issue, we add agents to a cluster
after every step of the model. If the absolute value of the difference between
an agent's opinion and the average opinion of a pre-existing cluster is within
a threshold (0.05), the agent is added to that cluster. If this is not the
case, the agent is added to a new cluster in which it is the first occupant.

